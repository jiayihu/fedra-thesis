%**************************************************************
% file contenente le impostazioni della tesi
%**************************************************************

%**************************************************************
% Frontespizio
%**************************************************************

% Autore
\newcommand{\myName}{Giovanni Jiayi Hu}                                    
\newcommand{\myTitle}{The continuum of computing}

% Tipo di tesi                   
\newcommand{\myDegree}{Tesi di laurea magistrale}

% Università             
\newcommand{\myUni}{Università degli Studi di Padova}

% Facoltà       
\newcommand{\myFaculty}{Corso di Laurea in Informatica}

% Dipartimento
\newcommand{\myDepartment}{Dipartimento di Matematica "Tullio Levi-Civita"}

% Titolo del relatore
\newcommand{\profTitle}{Prof.}

% Relatore
\newcommand{\myProf}{Tullio Vardanega}

% Luogo
\newcommand{\myLocation}{Padova}

% Anno accademico
\newcommand{\myAA}{2020-2021}

% Data discussione
\newcommand{\myTime}{Aprile 2021}


%**************************************************************
% Impostazioni di impaginazione
% see: http://wwwcdf.pd.infn.it/AppuntiLinux/a2547.htm
%**************************************************************

\setlength{\parindent}{0pt}   % larghezza rientro della prima riga
\setlength{\parskip}{0pt}   % distanza tra i paragrafi

% \setlength\bibitemsep{1.5\itemsep} % spazio tra entry


%**************************************************************
% Impostazioni di caption
%**************************************************************
\captionsetup{
    tableposition=top,
    figureposition=bottom,
    font=small,
    format=hang,
    labelfont=bf
}


%**************************************************************
% Impostazioni di graphicx
%**************************************************************
\graphicspath{{figures/}} % cartella dove sono riposte le immagini


%**************************************************************
% Impostazioni di hyperref
%**************************************************************
\hypersetup{
    %hyperfootnotes=false,
    %pdfpagelabels,
    %draft,	% = elimina tutti i link (utile per stampe in bianco e nero)
    colorlinks=true,
    linktocpage=true,
    pdfstartpage=1,
    pdfstartview=FitV,
    % decommenta la riga seguente per avere link in nero (per esempio per la stampa in bianco e nero)
    %colorlinks=false, linktocpage=false, pdfborder={0 0 0}, pdfstartpage=1, pdfstartview=FitV,
    breaklinks=true,
    pdfpagemode=UseNone,
    pageanchor=true,
    pdfpagemode=UseOutlines,
    plainpages=false,
    bookmarksnumbered,
    bookmarksopen=true,
    bookmarksopenlevel=1,
    hypertexnames=true,
    pdfhighlight=/O,
    %nesting=true,
    %frenchlinks,
    urlcolor=RoyalBlue,
    linkcolor=RoyalBlue,
    citecolor=RoyalBlue,
    %pagecolor=RoyalBlue,
    %urlcolor=Black, linkcolor=Black, citecolor=Black, %pagecolor=Black,
    pdftitle={\myTitle},
    pdfauthor={\textcopyright\ \myName, \myUni, \myFaculty},
    pdfsubject={},
    pdfkeywords={},
    pdfcreator={pdfLaTeX},
    pdfproducer={LaTeX}
}

%**************************************************************
% Impostazioni di itemize
%**************************************************************
\renewcommand{\labelitemi}{$\ast$}

%\renewcommand{\labelitemi}{$\bullet$}
%\renewcommand{\labelitemii}{$\cdot$}
%\renewcommand{\labelitemiii}{$\diamond$}
%\renewcommand{\labelitemiv}{$\ast$}


%**************************************************************
% Impostazioni di listings
%**************************************************************
\lstset{
    language=[LaTeX]Tex,%C++,
    keywordstyle=\color{RoyalBlue}, %\bfseries,
    basicstyle=\footnotesize\ttfamily,
    %identifierstyle=\color{NavyBlue},
    commentstyle=\color{Green}\ttfamily,
    stringstyle=\rmfamily,
    numbers=none, %left,%
    numberstyle=\scriptsize, %\tiny
    stepnumber=5,
    numbersep=8pt,
    showstringspaces=false,
    breaklines=true,
    frameround=ftff,
    frame=single
} 


%**************************************************************
% Impostazioni di xcolor
%**************************************************************
\definecolor{webgreen}{rgb}{0,.5,0}
\definecolor{webbrown}{rgb}{.6,0,0}

%**************************************************************
% Environment per ``namespace description''
%**************************************************************

\newenvironment{namespacedesc}{
    \vspace{10pt}
    \par \noindent                              % start new paragraph
    \begin{description} 
}{
    \end{description}
    \medskip
}

\newcommand{\classdesc}[2]{\item[\textbf{#1:}] #2}

%**************************************************************
% Nuovi comandi custom
%**************************************************************

\newcommand\blankpage{
    \null
    \thispagestyle{empty}
    \addtocounter{page}{-1}
    \newpage
    \null
    \thispagestyle{empty}
    \addtocounter{page}{-1}
    \newpage}
