\cleardoublepage
\phantomsection
\pdfbookmark{Abstract}{Abstract}
\begingroup

\chapter*{Abstract}

The Internet has evolved significantly since its inception. It has grown into a ubiquitous platform for everyday services from just a simple communication layer for information sharing between researchers. 

During the past decade, the sharing of server and networking capabilities - known as the cloud computing paradigm - has become a reality, giving users and companies access to virtually unlimited amounts of storage and computing power. Nowadays, there is an immense amount of constantly connected mobile devices, servers, and network components that offer their virtualised capabilities to their users.

Besides, as connectivity improved dramatically in affordability, bandwidth, reliability and reachability, people started bringing their devices with them and accessing the Internet anywhere and anytime. This in turns, boosted the evolution of mobile computing and led to the emergence of richer client-side web applications.  Nowadays, we are amid the so-called Internet of Things (IoT), where everyday objects (things) are connected to the Internet and each other. 

Thus, it is natural that as consumers, we want our Internet-capable devices (e.g., mobile phones, thermostats, electric vehicle) to adapt seamlessly to our changing lives and expectations, regardless of location.

This thesis presents an information-rich pervasive computational continuum that seamlessly combines this data and computing power to model, manage, control and make use of virtually any realisable sub-system of interest.  The computing continuum is a ubiquitous system where distributed resources and services on the whole computing continuum are dynamically aggregated on-demand to support different ranging services.


%\vfill
%
%\selectlanguage{english}
%\pdfbookmark{Abstract}{Abstract}
%\chapter*{Abstract}
%
%\selectlanguage{italian}

\endgroup			

\vfill

